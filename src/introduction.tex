\section{Introduction}

Thanks to the computer hardware advances in recent years and the consequent advances in artificial intelligence, robots are supposed by society to help people in a lot of different tasks.
They are supposed to work in changing conditions for long periods and replace humans not only in the repetitive tasks but also in the tasks where humans can fail due to their emotional changes, due to their slow reactions or low dexterity.
Robots are also supposed to be helpful in the tasks, where long-term concentration is needed or where the low frequency of actions is in contrast to the necessity of rapid and precise reaction.
Moreover, due to the massive data flow and progress in the databases, autonomous systems are now understood as knowledge holders that can help inexperienced humans to reason in the unusual situations or to chose the best solution before more experienced human arrives.

Autonomous robots have to move through the environment populated by humans without causing chaos, interrupting people from their tasks, and, on the other side, be at the right time in the right place.
Therefore it is necessary to understand principles and changes in the human populated areas.
We can see on the ongoing experiments, that robots able to predict human behaviour can provide better services in the meaning of quality and also quantity[STRANDS].
It is obvious [KUCNER], that understanding of the flow of moving people, that is usually not directly specified by some external rule, but evince some internal rules evolving in the time, leads to the faster moving through the highly populated area.
The understanding of the usual human behaviour also leads to possible anomalous behaviour detection, that can start some particular robotic task, such as offer help to the lost human [RAKOUSKO] or alarm security [MESAS].
