\section{Introduction}

Thanks to the computer hardware advances in recent years and to the consequational advances in artificial intelligence, robots are supposed by society to help people in lot of different tasks.
They are supposed to work in changing conditions for long periods of time and replace humans not only in the repetitive tasks but also in the tasks where humans can fail due to their emotional changes, due to their slow reactions or low dexterity.
Robots are also supposed to be helpful in the tasks, where long-term concentration is needed or where the low frequency of actions is in the contrast to the necessity of rapid and precise reaction.  
Moreover, due to the massive dataflow and progress in the databases, autonomous systems are now understood as knowledge holders that can help unexpierenced humans to reason in the unussual situations or to chose best solution before more experienced human arrive.

To be realy helpful, autonomous robots have to move through human populated environment without causing chaos, interrupting people from their tasks, and, on the other side, be in the right time on the right place.
Therefor it is necessary to understand principles and changes in the human populated areas.
We can see on the ongoing experiments, that robots able to predict human behavior can provide better services in the meaning of quality and also quantity[STRANDS].
It is obvious [KUCNER], that understanding of the flow of moving people, that is usualy not directly specified by some external rule, but exposes(?) some inner rules evolving in the time, leads to the faster moving through highly populated area.
The understanding of the usual human behavior also leads to possible anomal behavior detection, that can start some special robotic task, such as offer a help to the lost human [RAKOUSKO] or to alarm security [MESAS].

