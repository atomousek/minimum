\section{Introduction}

Thanks to the computer hardware advances in recent years and the consequent advances in artificial intelligence, robots are expected by society to help people in a lot of different tasks.
They are supposed to work in changing conditions for long periods and replace humans not only in the repetitive tasks but also in the tasks where humans fail due to their emotional nature, due to their slow reactions or low dexterity.
Robots are also supposed to be helpful in the tasks, where long-term concentration is needed or where the low frequency of actions is in contrast to the necessity of rapid and precise reaction.
Moreover, due to the massive data flow and progress in the databases, autonomous systems are now understood as knowledge providers that can help inexperienced humans to reason in the unusual situations or to chose the best solution before more experienced human arrives.

Nowadays, robots can operate very effectively in controlled and precisely mapped environments.
However, the current aim is to make them work in the general environment for a long time.
Such a general environment evince various irregularities - changes over time as a result of natural processes or human activity.
These changes of the environment lead to the gradual worsening of the static maps, and it is not possible only to change the obsolete static map with the new one \cite{Dayoub2009adaptive}.
The explicit representation of these changes over time in the maps leads not only to the possibility of performing the autonomous tasks for a longer time 
%\cite{biber2009experimental}, \cite{tipaldi2013lifelong}, \cite{kucner2013conditional}, \cite{krajnik2017fremen}, \cite{churchill2013experience}, \cite{konolige2009towards}, \cite{hochdorfer2009towards}
\cite{biber2009experimental,tipaldi2013lifelong,kucner2013conditional,krajnik2017fremen,churchill2013experience,konolige2009towards,hochdorfer2009towards}
 but also to the better performance 
%\cite{hawes2017strands}, \cite{santos2016lifelong}, \cite{kunze2018artificial}, \cite{santos2017spatio}, \cite{hanheide2017and}.
\cite{hawes2017strands,santos2016lifelong,kunze2018artificial,santos2017spatio,hanheide2017and}.

As it is common in the robotics to model the environment using an occupancy grid \cite{elfes1989using}, it is natural to model changes of the environment over the grid.
In \cite{kucner2013conditional}, the authors model the typical direction of change in every cell based on the previous and next position of the measured phenomenon in the neighbour of the cell. 
Another discrete short-term model can be found in \cite{wang2014modeling}, where authors predict the path of the measured phenomenon based on the actual situation in the grid using the input-output Markov model.
Apart from modelling the directions of the movement of particles in the environment, the robotic community focuses on modelling the changes in the environment over time.
A very popular approach to model time-dependent phenomena is to create seasonal windows (usually one day long) and to model the environment over the different parts of these windows \cite{Van2008Using,Blanke2009Daily}.
This approach models patterns of the environment change, but it is necessary to define the principal periodicity, i.e., the length and the resolutions of windows.
The use of the time intervals suffers from the discontinuity at the borders of the intervals, and there is the necessity to overcome this discontinuity \cite{chinellato2017incremental}.
To surmount these issues, authors of \cite{krajnik2017fremen} suggest to apply the spectral analysis of the time domain on every cell in the occupancy grid. 
They were able to model the long-term changes over the time \cite{Krajnik2014Longterm} in the 3d map\cite{Krajnik2014Froctomap}.
The authors then extended this spatio-temporal model to be able to predict also the direction of the movement through the cell in the specific time \cite{molina2018modelling}.
The main problem of this approach is the spatial independence of the neighbouring cells which can be addressed by spatial ordering \cite{Cliff1975Model}.
Nevertheless, the spatial ordering has to be predefined a priory and cannot be changed during the learning process \cite{Shi2018Machine}.
The necessity to model spatial dependencies in the dynamic environment leads to the idea of continuous representation of the environment.
Although the continuous models of the environment changes are computationally demanding, they are beneficial due to their memory efficiency\cite{o2012gaussian}[MESAS].
Moreover, using specific optimisations may be applied \cite{ramos2016hilbert}.

In \cite{O'Callaghan2011Learning} authors propose to use classical navigational techniques over the occupancy grid, such as potential fields, to define the prior directional map, update this map using detected human motions, and transform this map into the continuous map using the Gaussian processes.
This approach suffers from averaging angles, which can lead to the unwanted resulting angles in cases when humans traverse some position in both directions.
In the later work \cite{McCalman2013MultiModal} they improved the update to be able to model multimodal distributions of directions. 
The authors of \cite{kucner2013conditional} also changed their approach of modelling directions and proposed a continuous map \cite{Kucner2016Tell}.  
To model speed and direction of people and the flow of the wind  \cite{kucner2017enabling}, they introduced the expectation-maximisation method based on the Independent von Mises–Gaussian distribution \cite{roy2012mixture}.
They also showed that the model of the movement of people could be used to effective navigation of the robot through the crowd \cite{palmieri2017kinodynamic}.
The flow of continuous media was also modelled in \cite{Guizilini2015Nonparametric} by applying Gaussian processes \cite{rasmussen2004gaussian} to the measured data.
To model the periodical nature of the modelled phenomena, authors used frequency analysis to divide the dataset into the different subsamples with different periodical properties and modelled them separately.
For predicting the influenza epidemic over the United States, which is a highly periodic event with the periodicity of one year, authors \cite{Senanayake2016Predicting} used a specific combination of kernels, one of them was \textit{periodic kernel}.
Periodic kernel \cite{Tompkins2018Fourier} and its multidimensional variant \cite{Tompkins2018Index} were then optimized in the sense of formerly proposed Hilbert maps \cite{ramos2016hilbert}.
Hilbert maps were subsequently improved to be updated incrementally \cite{Senanayake2017Bayesian}.

To operate in the dynamic human-populated environments, a robot has to build and update the model of its environment during its standard activity.
This can be done in an active way, by visiting locations unrelated to an actually performed task, simply for the purpose of data collection.
An activity, when a robot decides itself where to go to collect data, is referred to as robotic exploration.
The first ever life-long spatio-temporal exploration was designed in \cite{Krajnik2015Lifelong}.
The authors used their method FreMEn \cite{krajnik2017fremen} to continuously build the map.
They research different exploration strategies and define the fundamental issue, which they call \textit{exploration versus exploitation dilemma} \cite{kulich2016explore}.
Authors of \cite{Molina2019Go} used a similar approach to exploration as in \cite{Krajnik2015Lifelong} and built the spatio-temporal model of directions and velocities of moving humans.
In \cite{Duckworth2016Unsupervised}, the robot during its task performance classified different types of human behaviour in the office by clustering partially observed trajectories of moving humans.
The operating robot was then able to predict human position in the near future when it encountered him and consequently avoid the occupied places. 



In summary, the periodicities of the human habits (sleep, going to work) are dominant compared to progress or evolution in the horizon of months and years, and therefore we neglect other temporal influences in the model.
To model spatial dependencies, continuous models outperform the discrete ones in term of representing a fidelity memory efficiency, which is a necessary requirement for the ability to store large spatio-temporal maps. 
Inspired by these ideas and recent advances in modelling dynamic worlds, we created a projection of the time for the long-term models of the changes in the human-populated environment.
To ensure the continuity on the borders of the time intervals induced by the natural periodicities, we project the time into a set of circles, where every circle is derived from the periodicity detected in the measured phenomenon.
The time-dependent events (that match the periodicity) projected into the corresponding circle thus appear on the similar positions of that circle.
This projection reflects our concept that human behaviour in the morning is similar at the different days, the temperature in the night is lower than in the day, and that the situation changes insignificantly during the midnight contrary to the abrupt change of the day number.
The measured phenomenon projected into this vector space can be then analysed using standard statistical and machine learning tools.

The ultimate goal of this dissertation thesis is to create a general tool to model spatio-temporal phenomena useful for robots operating in large dynamic areas for long periods of time.
