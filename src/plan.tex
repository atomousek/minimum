\section{Research objectives and dissertation plan}

This research aims to create a tool based on the warped hypertime projection for the complex control of the autonomous robot in the dynamic, human populated environment.
Such an environment includes changes in 
\begin{itemize}
    \item the structure derived from open or closed doors,
    \item the number of presented people,
    \item the visual conditions due to the natural cycles,
    \item the dominant directions of flow of people,
    \item the occupancy of different areas by people, and so on.
\end{itemize}
The prediction of different phenomena and adaption of the robot to the actual state leads to the better performance of the accomplishing of robot tasks \cite{hawes2017strands}.
However, different predictions of different variables based on the different scenarios lead to different machine learning tasks.
Therefore we want to define the ways, how to use different classes of machine learning methods using warped hypertime:
\begin{enumerate}
    \item Frequency Analysis
    \begin{itemize}
        \item Hyper Time (Space) with chosen resolution creation (sec. \ref{sec:resolution}),
        \item the frequency of measurements function (\ref{eqn:distribution}),
        \item with missing values handling (\ref{eqn:cluster_weights}), (\ref{eq:components}),
    \end{itemize}
    \item Binary Classification
    \begin{itemize}
        \item Hyper Time (Space) creation (sec. \ref{sec:whyte}, \ref{sec:whytes}),
        \item suitable method for binary classification,
        \item including outliers handling [mesas] (Figure \ref{graph:mathew90}),
    \end{itemize}
    \item Multiclass Classification
    \begin{itemize}
        \item Hyper Time (Space) creation,
        \item suitable method for multiclass classification,
        \item with outliers handling,
    \end{itemize}
    \item Regression
    \begin{itemize}
        \item Hyper Time (Space) creation (sec. \ref{sec:whyte}, \ref{sec:whytes}),
        \item suitable method for regression,
        \item with outlier handling.
    \end{itemize}
\end{enumerate}

We also plan to collect a set of datasets of different phenomena to apply these methods, including:
\begin{itemize}
    \item the long term spatial human detection on the 24/7 basis (in progress),
    \item the navigation maps in changing outdoor environment (in progress),
    \item the long term spatio-directional human detections [MALL, pripadne KUCNER],
    \item the long term person presence locations \cite{krajnik2015s},
    \item and the long term open/closed door detection \cite{krajnik2014long}.
\end{itemize} 
    
        

