\section{Research Objectives and Dissertation Plan}

This research aims to create a universal tool based on the warped hypertime projection for the complex control of the autonomous robot in dynamic, human-populated environments.
Such an environment includes changes in 
\begin{itemize}
    \item the structure derived from open or closed doors,
    \item the number of people present,
    \item the visual appearance due to the natural cycles,
    \item the dominant directions of pedestrian flow,
    \item the occupancy of different areas by people, and so on.
\end{itemize}
The prediction of different phenomena and adaptation of the robot to the actual state leads to better performance of the accomplishing of robot tasks \cite{hawes2017strands}.
However, different predictions of different variables based on the different scenarios lead to different machine learning tasks.
Therefore we want to define the ways, how to use different classes of machine learning methods using warped hypertime:
\begin{enumerate}
    \item Frequency Analysis
    \begin{itemize}
        \item Hyper Time (Space) with chosen resolution creation (sec. \ref{sec:resolution}),
        \item the frequency of measurements function (\ref{eqn:distribution}),
        \item handling missing values (\ref{eq:components}), (\ref{eq:probabilityWHyTe}), (\ref{eqn:clusterWeights}),
    \end{itemize}
    \item Binary Classification
    \begin{itemize}
        \item Hyper Time (Space) creation (sec. \ref{sec:whyte}, \ref{sec:whytes}),
        \item suitable method for binary classification,
        \item including outliers detection \cite{Vintr2018Spatiotemporal} (Figure \ref{graph:mathew90}),
    \end{itemize}
    \item Multiclass Classification
    \begin{itemize}
        \item Hyper Time (Space) creation,
        \item suitable method for multiclass classification,
        \item with possible outliers detection,
    \end{itemize}
    \item Regression
    \begin{itemize}
        \item Hyper Time (Space) creation (sec. \ref{sec:whyte}, \ref{sec:whytes}),
        \item suitable method for regression,
        \item with possible outlier detection.
    \end{itemize}
\end{enumerate}

We also plan to collect a set of datasets of different phenomena to apply these methods, including:
\begin{itemize}
    \item the long term spatial human detection on the 24/7 basis (in progress),
    \item the navigation maps in changing outdoor environment (in progress),
    \item the long term spatio-directional human detections (in progress),
    \item the long term person presence locations \cite{krajnik2015s},
    \item and the long term open/closed door detection \cite{krajnik2014long}.
\end{itemize} 
    
Finally, we want to deploy warped hypertime based methods on a real robot.
        

