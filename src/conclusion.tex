\section{Conclusion\label{sec:conclusion}}

This doctoral thesis proposal summarises the latest advances in modelling the dynamics of the robot operational area and present the novel approach, which addresses the issues, that robot face when operating in human-populated environments.
The main idea of the proposed approach lies in the projection of time into the bounded subset of the multidimensional vector space.
The projection and the dimensionality of the vector space are derived from the statistical characteristic of the measured phenomenon.
It is possible to apply usual machine learning methods to infer the spatio-temporal information from the projected measurements.
It was verified on different real-life datasets, that mined information is sufficient to predict human presence, robot speed across dynamic environments, the visual appearance of several scenes over time, and behaviour of a human crowd.
These predictions were used to detect anomalies in human behaviour.
It was shown in the experiments, that the predictive power of the proposed approach is comparable or exceed the state-of-the-art methods and it performs well in more complex scenarios.
The integral part of this work is the collection of the long-term datasets.