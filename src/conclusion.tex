\section{Conclusion\label{sec:conclusion}}

This doctoral thesis proposal summarises the latest advances in modelling the dynamics of the robot operational area and present the novel approach, which addresses the known issues.
The main idea of the proposed approach lies in the projection of time into the bounded subset of the multidimensional vector space.
The projection and the dimensionality of the vector space are derived from the statistical characteristic of the measured phenomenon.
It is possible to apply usual machine learning methods to mine the spatio-temporal information from the projected measurements and it was verified on different real-life datasets, that mined information is sufficient to predict a human behaviour in his natural environment, a duration of robot traversal times across the dynamic environment, visible points of interest in the specific times, behaviour of a human crowd, and to detect anomalies in a human behaviour.
It was shown in the experiments, that the predictive power of the proposed approach is comparable or exceed the state-of-the-art methods and it excels in more complex scenarios.
The integral part of this work is the collection of the long-term datasets.